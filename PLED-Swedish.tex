\documentclass{article}
\usepackage[swedish]{babel}
\usepackage[a4paper]{geometry}
%\usepackage{amsmath}
\usepackage{graphicx}
\usepackage[export]{adjustbox}
%\usepackage[colorlinks=true, allcolors=blue]{hyperref}

\setlength\parindent{15pt}
\begin{document}
	\begin{figure}[t]
		\vspace{-60pt}
		\includegraphics[width= 32 mm, center]{SRDDark.png}
	\end{figure}
	
	\title{\vspace{-30pt}Beslutsmöte, Programledningen C/D}
	\author{\textbf{Annie Predel}, Programledningsrepresentant}
	\date{23 Oktober, 2024}
	
	\maketitle
	
%	\section{Sammanfattning}
	
	
	
	\section{Ändringar}
	%\href{link}{name}
	\subsection{Kursutbud}
	EDAP25 Distribuerade System är periodiserat men det gavs inte ut på två år. Orsaken är problem med att anställa en ny lärare.\newline
	EDAN01 Contrstraint Programmering läggs ned av samma orsak. Inget PLED kan göra åt det eftersom det är LTHs anställningsprocess som är dåligt.\newline
	EDAA45 Programmering, grundkurs och EDAA60 Datorer och Datoranvändning sammanställs till en enda 10.5HP kurs. Forslag har skickats till institutionen att ha EDAA60 kvar som kurskod ifall det behövs för de som är antagna till senare del.\newline
	EITA65 Digitalisering Läggs ner och ersätts med en 12HP kurs med samma innehåll förutom det som fanns i EDAA60.\newline
	Allt annat godkänds, se föregående protokoll.
	
	\subsection{Delmoment}
	EITA55 Kommunikationssystem ändrar fördelning av HP gällande moduler.\newline
	MAMF50 Användbarhetsutvärdering sammanställer två moduler.\newline
	MAMN25 Interaktionsdesign har en ny modul med redistributerade HP till modulerna.
	
	\subsection{Aktuell information}
	
	EDAA75 och EDAA40 Diskreta Strukturer har ny kursansvarig Susanna Rezende är nu ansvarig för kursen Diskreta Strukturer. Det föreslågs att förena de två kurserna till en enda 7.5HP kurs för att underlätta för läraren.\newline
	Utvärdering av utbildning har pågått. Representanter från Chalmers, KTH och Mitt universitet har pratat med studenterna om hur utbildning går till, kulturen och vad man vill förbättra.\newline
	Programledningen kommer kolla över strul i rättningen på EDAN40 Funktionsprogrammerings tentor.\newline
	Nästa möte är på den 28:e November för Lär- och timplan.\newline
	PLED vill ha lunch med SRD för att prata om vad som helst.\newline	
	SRD kan be kursansvarig att ha CEQ möte även om det inte finns årskursrepresentant eller kursombud, med hänsyn på kurser i År 4 och 5 och kurser med problem såsom Diskreta Strukturer och Funktionsprogrammering.\newline
	PLED tänker flytta Datateknik mer åt mjukvaru-hållet, vill ha input av SRD. De föreslåg Databaser, operativsystem och/eller AI i grundblocket.\newline
	Mikolaj föreslåg en workshop för specialiseringar i samband med lunch. Troligtvis i Februari.\newline
	Det föreslågs en enkät om kurserna i grundblocket: Vad saknas, vad vill man ta bort. Kurser som krävs enligt LTH och högskoleförordningen kan inte tas bort och det skulle specificeras.\newline
	PLED instämmer att det skulle vara möjligt för SRD att ha en kort presentation om kursombud inför varje kurs.\newline
	PLED kommer ha interna val och kan bytas ut nästa år i Juni-Augusti.
	
	
\end{document}